\documentclass[10pt, letterpaper]{article}

% Packages:
\usepackage[
    ignoreheadfoot, % set margins without considering header and footer
    top=2 cm, % seperation between body and page edge from the top
    bottom=2 cm, % seperation between body and page edge from the bottom
    left=2 cm, % seperation between body and page edge from the left
    right=2 cm, % seperation between body and page edge from the right
    footskip=1.0 cm, % seperation between body and footer
    % showframe % for debugging 
]{geometry} % for adjusting page geometry
\usepackage{titlesec} % for customizing section titles
\usepackage{tabularx} % for making tables with fixed width columns
\usepackage{array} % tabularx requires this
\usepackage[dvipsnames]{xcolor} % for coloring text
\definecolor{primaryColor}{RGB}{0, 0, 0} % define primary color
\usepackage{enumitem} % for customizing lists
\usepackage{fontawesome5} % for using icons
\usepackage{amsmath} % for math
\usepackage[
    pdftitle={Arman Sheikhhosseini's CV},
    pdfauthor={Arman Sheikhhosseini},
    pdfcreator={LaTeX with RenderCV},
    colorlinks=true,
    urlcolor=primaryColor
]{hyperref} % for links, metadata and bookmarks
\usepackage[pscoord]{eso-pic} % for floating text on the page
\usepackage{calc} % for calculating lengths
\usepackage{bookmark} % for bookmarks
\usepackage{lastpage} % for getting the total number of pages
\usepackage{changepage} % for one column entries (adjustwidth environment)
\usepackage{paracol} % for two and three column entries
\usepackage{ifthen} % for conditional statements
\usepackage{needspace} % for avoiding page brake right after the section title
\usepackage{iftex} % check if engine is pdflatex, xetex or luatex

% Ensure that generate pdf is machine readable/ATS parsable:
\ifPDFTeX
    \input{glyphtounicode}
    \pdfgentounicode=1
    \usepackage[T1]{fontenc}
    \usepackage[utf8]{inputenc}
    \usepackage{lmodern}
\fi

\usepackage{charter}

% Some settings:
\raggedright
\AtBeginEnvironment{adjustwidth}{\partopsep0pt} % remove space before adjustwidth environment
\pagestyle{empty} % no header or footer
\setcounter{secnumdepth}{0} % no section numbering
\setlength{\parindent}{0pt} % no indentation
\setlength{\topskip}{0pt} % no top skip
\setlength{\columnsep}{0.15cm} % set column seperation
\pagenumbering{gobble} % no page numbering

\titleformat{\section}{\needspace{4\baselineskip}\bfseries\large}{}{0pt}{}[\vspace{1pt}\titlerule]

\titlespacing{\section}{
    % left space:
    -1pt
}{
    % top space:
    0.3 cm
}{
    % bottom space:
    0.2 cm
} % section title spacing

\renewcommand\labelitemi{$\vcenter{\hbox{\small$\bullet$}}$} % custom bullet points
\newenvironment{highlights}{
    \begin{itemize}[
        topsep=0.10 cm,
        parsep=0.10 cm,
        partopsep=0pt,
        itemsep=0pt,
        leftmargin=0 cm + 10pt
    ]
}{
    \end{itemize}
} % new environment for highlights


\newenvironment{highlightsforbulletentries}{
    \begin{itemize}[
        topsep=0.10 cm,
        parsep=0.10 cm,
        partopsep=0pt,
        itemsep=0pt,
        leftmargin=10pt
    ]
}{
    \end{itemize}
} % new environment for highlights for bullet entries

\newenvironment{onecolentry}{
    \begin{adjustwidth}{
        0 cm + 0.00001 cm
    }{
        0 cm + 0.00001 cm
    }
}{
    \end{adjustwidth}
} % new environment for one column entries

\newenvironment{twocolentry}[2][]{
    \onecolentry
    \def\secondColumn{#2}
    \setcolumnwidth{\fill, 4.5 cm}
    \begin{paracol}{2}
}{
    \switchcolumn \raggedleft \secondColumn
    \end{paracol}
    \endonecolentry
} % new environment for two column entries

\newenvironment{threecolentry}[3][]{
    \onecolentry
    \def\thirdColumn{#3}
    \setcolumnwidth{, \fill, 4.5 cm}
    \begin{paracol}{3}
    {\raggedright #2} \switchcolumn
}{
    \switchcolumn \raggedleft \thirdColumn
    \end{paracol}
    \endonecolentry
} % new environment for three column entries

\newenvironment{header}{
    \setlength{\topsep}{0pt}\par\kern\topsep\centering\linespread{1.5}
}{
    \par\kern\topsep
} % new environment for the header

\newcommand{\placelastupdatedtext}{% \placetextbox{<horizontal pos>}{<vertical pos>}{<stuff>}
  \AddToShipoutPictureFG*{% Add <stuff> to current page foreground
    \put(
        \LenToUnit{\paperwidth-2 cm-0 cm+0.05cm},
        \LenToUnit{\paperheight-1.0 cm}
    ){\vtop{{\null}\makebox[0pt][c]{
    \small\color{gray}\textit{Last updated in September 2025}\hspace{\widthof{Last updated in September 2025}}
    }}}%
  }%
}%

% save the original href command in a new command:
\let\hrefWithoutArrow\href

% new command for external links:


\begin{document}
    \newcommand{\AND}{\unskip
        \cleaders\copy\ANDbox\hskip\wd\ANDbox
        \ignorespaces
    }
    \newsavebox\ANDbox
    \sbox\ANDbox{$|$}

            % Optional last updated text overlay
            \placelastupdatedtext

            % Header
            \begin{header}
                \fontsize{25 pt}{25 pt}\selectfont Arman Sheikhhosseini

                \vspace{3 pt}
                {\Large\bfseries DevOps Engineer}
                \vspace{3 pt}

                \normalsize
                \mbox{Turin, Italy}%
                \kern 5.0 pt%
                \AND%
                \kern 5.0 pt%
                \mbox{\hrefWithoutArrow{mailto:arman.sheikhhosseini@gmail.com}{arman.sheikhhosseini@gmail.com}}%
                \kern 5.0 pt%
                \AND%
                \kern 5.0 pt%
                \mbox{\hrefWithoutArrow{tel:+39-379-268-2755}{+39\,379\,268\,2755}}%
                \kern 5.0 pt%
                \AND%
                \kern 5.0 pt%
                \mbox{\hrefWithoutArrow{https://armansheikhhosseini.github.io/}{armansheikhhosseini.github.io}}%
                \kern 5.0 pt%
                \AND%
                \kern 5.0 pt%
                \mbox{\hrefWithoutArrow{https://www.linkedin.com/in/thearmansh/}{linkedin.com/in/thearmansh}}%
                \kern 5.0 pt%
                \AND%
                \kern 5.0 pt%

            \end{header}

            % spacing between header and content
            \vspace{0.3 cm}


            \section{Professional Summary}
                \begin{onecolentry}
                    DevOps engineer focused on reliable, scalable AWS platforms with Infrastructure as Code (Terraform, Ansible), containers (Docker, Kubernetes), and CI/CD (Jenkins, GitLab). Delivered 60\% faster provisioning, 25\% lower MTTR via Prometheus/Grafana observability, and sustained 99.9\% uptime supported by automated backups and disciplined incident response. Strong Bash/Python automation with a networking foundation from NOC operations; security-minded and currently an M.Sc. Cybersecurity candidate at Politecnico di Torino.
                \end{onecolentry}

            \section{Experience}
                % Torob
                \begin{twocolentry}{Nov 2022 -- Present}
                    \textbf{DevOps Engineer}, Torob
                \end{twocolentry}
                \vspace{0.10 cm}
                \begin{onecolentry}
                    \begin{highlights}
                        \item Provisioned and deployed highly available AWS infrastructure (EC2, VPC, IAM, Redshift, EBS, RDS, Route~53) using Terraform, enhancing scalability and reliability while reducing provisioning time by 60\%.
                        \item Implemented automated backup and recovery processes utilizing AWS Backup and Amazon S3 versioning, ensuring data integrity and availability in compliance with industry best practices.
                        \item Deployed and automated configuration management using Ansible, streamlining setup across environments and eliminating repetitive manual tasks.
                        \item Optimized system monitoring and alerting with Prometheus and Grafana dashboards, achieving a 25\% reduction in mean time to resolution (MTTR) and enabling proactive incident handling.
                        \item Automated daily infrastructure tasks using Rundeck, encompassing AWS service management and backups, thereby improving operational efficiency and consistency.
                        \item Containerized services with Docker, accelerating deployment cycles and simplifying environment management.
                        \item Developed scripts in Bash and Python for system automation, reducing manual errors and saving developer hours.
                        \item Administered Jenkins CI/CD pipelines, performing routine maintenance, updates, configuration, and cloud configuration tasks.
                    \end{highlights}
                \end{onecolentry}

                \vspace{0.2 cm}

                % Shatel - NOC Engineer
                \begin{twocolentry}{Jul 2020 -- Sep 2022}
                    \textbf{Network Operations Center Engineer}, Shatel
                \end{twocolentry}
                \vspace{0.10 cm}
                \begin{onecolentry}
                    \begin{highlights}
                        \item Utilized PRTG and Zabbix to monitor network performance, achieving a 95\% proactive issue resolution rate before end-user impact.
                        \item Configured and optimized routers to ensure 99.9\% network uptime, enhancing overall service reliability.
                        \item Employed SolarWinds for in-depth network traffic analysis, identifying and mitigating bottlenecks, leading to a 30\% reduction in downtime.
                        \item Coordinated with cross-functional teams to resolve network incidents promptly, minimizing service disruptions and maintaining customer satisfaction.
                        \item Maintained comprehensive network documentation and incident reports, ensuring compliance with internal standards and facilitating continuous improvement.
                    \end{highlights}
                \end{onecolentry}

                \vspace{0.2 cm}

                % Shatel - Technical Support Specialist
                \begin{twocolentry}{Jun 2018 -- Jul 2020}
                    \textbf{Technical Support Specialist}, Shatel
                \end{twocolentry}
                \vspace{0.10 cm}
                \begin{onecolentry}
                    \begin{highlights}
                        \item Resolved an average of 50+ technical support tickets per week, maintaining a 95\% customer satisfaction rating.
                        \item Conducted regular system and software updates, preventing potential downtimes and enhancing system performance by 20\%.
                        \item Collaborated with cross-functional teams to identify and resolve recurring technical issues, leading to a 30\% reduction in repeat tickets.
                        \item Maintained comprehensive documentation of technical issues and resolutions, streamlining knowledge sharing and reducing resolution time by 15\%.
                        \item Utilized remote support tools like Zendesk and TeamViewer to troubleshoot and resolve customer issues efficiently.
                    \end{highlights}
                \end{onecolentry}

            \section{Education}
                \begin{twocolentry}{Sep 2024 -- Jul 2026}
                    \textbf{Politecnico di Torino}, Master's in Cybersecurity (Candidate)
                \end{twocolentry}

                \vspace{0.10 cm}
                \begin{twocolentry}{Sep 2018 -- Apr 2018}
                    \textbf{Azad University}, Bachelor of Computer Software Engineering
                \end{twocolentry}

            \section{Skills}
                \begin{onecolentry}
                    \textbf{Skills:} Cloud computing expertise; Web server management (Apache, Nginx); CI/CD practices (GitLab, Jenkins); Container orchestration (Docker, Kubernetes); Infrastructure as code (Terraform, Ansible); Scripting (Python, Bash); Monitoring solutions (CloudWatch, Prometheus, Grafana); AWS services (EC2, VPC, S3, RDS, Lambda); Unix/Linux administration.
                \end{onecolentry}

            \section{Certifications}
                \begin{onecolentry}
                    Cisco Certified Network Associate Security (CCNA) \;\;\; \textbullet\; Linux Professional Institute LPIC-1 \;\;\; \textbullet\; Linux Professional Institute LPIC-2 \\
                    Microsoft Certified Systems Administrator (MCSA) \;\;\; \textbullet\; CompTIA Security+ \;\;\; \textbullet\; Certified Kubernetes Administrator (CKA) \;\;\; \textbullet\; AWS Certified DevOps Engineer
                \end{onecolentry}

            \section{Languages}
                \begin{onecolentry}
                    English: Fluent \;\;\; \textbullet\; Italian: Intermediate
                \end{onecolentry}

        \end{document}